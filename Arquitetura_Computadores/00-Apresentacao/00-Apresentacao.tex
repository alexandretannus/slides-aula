\documentclass[aspectratio=169,
				xcolor=table]{beamer}

% Load general definitions
\usepackage[utf8]{inputenc}
%\usepackage[T1]{fontenc}
\usepackage[brazil]{babel}
\usepackage{amsmath}
\usepackage{amsfonts}
\usepackage{amssymb}
\usepackage{graphicx}
\usepackage{verbatim}
\usepackage{cancel}
\usepackage{askmaps}
\usepackage{tabularx}
\usepackage[table]{xcolor}
%\usepackage{tikz}
\usepackage{multirow}
\usepackage{mathtools}
\usepackage{color, colortbl}
\usepackage{etoolbox}
\usepackage{pbox}
\usepackage{changepage}
\usepackage{xpatch}
\usepackage{array}
\usepackage{marvosym}
\usepackage{tabu}
\usepackage{multicol}
\usepackage{listings}
\usepackage{underscore}
\usepackage{filecontents}
\usepackage[]{algorithm2e}
\usepackage{ragged2e}

\newcolumntype{P}[1]{>{\centering\arraybackslash}m{#1}}
\definecolor{Gray}{gray}{0.75}
\definecolor{Gray2}{gray}{0.85}

\definecolor{lightBlue}{HTML}{DAE8FC}
\definecolor{Blue}{RGB}{51, 51, 204}

%\useinnertheme[lily]{rounded}
\usetheme{UniEvangelica}
%\usetheme{Copenhagen}
%\usetheme{Berlin}
%\usecolortheme{dolphin}
\tolerance=1
\emergencystretch=\maxdimen
\hyphenpenalty=10000
\hbadness=10000

\setbeamertemplate{navigation symbols}{}%remove navigation symbols


\let\olditem=\item% 
\renewcommand{\item}{\olditem \justifying}%
\def\center{\trivlist \centering\item\relax}
\def\endcenter{\endtrivlist}

\setbeamertemplate{itemize/enumerate body begin}{\large}
\setbeamertemplate{itemize/enumerate subbody begin}{\large}

\setbeamertemplate{itemize item}{\raisebox{0.1ex}{$\blacktriangleright$}\hskip0.1em}
\setbeamertemplate{itemize subitem}{\raisebox{0.1ex}{$\blacktriangleright$}\hskip0.1em}

\newcommand{\greenarrow}{\textcolor{green}{\rotatebox[origin=c]{180}{\MVArrowDown}}}

\newcommand{\redarrow}{\textcolor{red}{\MVArrowDown}}

%\newcommand{\ftable}{
%	\begin{table}
%		\large
%		\centering
%		\rowcolors{1}{\ifnumless{\rownum}{2}{Blue}{lightBlue}}{}
%}

\newenvironment{eftable}{
	\begin{table}
		\large
		\centering
		\rowcolors{1}{}{Blue}
		\rowcolors{1}{\ifnumless{\rownum}{2}{Blue}{lightBlue}}{}
	}
	{
	\end{table}
}


%\setbeamertemplate{frametitle}
%{
%	%\vspace*{-2em}	
%	\insertframetitle
%
%	 %\textcolor{white}{\LARGE \insertframetitle}
%
%}

% Specific definitions
\institute[]{\uppercase{Engenharia de Software}}
\title[]{Arquitetura e Organização de Computadores}
\subtitle[]{\uppercase{Apresentação da Disciplina}}
\author[]{Prof. Alexandre Tannus}
\date{}

\AtBeginSection{\frame{\tableofcontents[currentsection]}}

\begin{document}
	\begin{frame}
		\titlepage
	\end{frame}
	
	\begin{frame}{Professor Alexandre Tannus}
		\begin{center}
			\centering
			\huge Bacharel em Engenharia da Computação
			\begin{figure}
			\centering
				\includegraphics[height=2cm, keepaspectratio]{../figs/puc.jpg} 			
			\end{figure}
			\huge Mestre em Engenharia Elétrica
			\begin{figure}
			\centering
				\includegraphics[height=2cm, keepaspectratio]{../figs/ufmg.png} 		
			\end{figure}
		\end{center}
	\end{frame}
	
	\begin{frame}[allowframebreaks]{Objetivos}
		\begin{itemize}
			\item Conhecer a evolução dos computadores e a estrutura dos computadores atuais (teoria).
			\vspace{1em}
			\item Assimilar os sistemas de numeração e contextualização da teoria nos sistemas decimal, binário e hexadecimal (teoria). 
			\vspace{1em}
			\item Compreender a execução de uma instrução pelo processador, abordando a memória, o ciclo de máquina, os dados e as instruções do programa.
			\vspace{1em}
			\item Reconhecer da arquitetura básica do processador Intel e algumas de suas instruções e utilização de uma ferramenta IDE (Masm) para linguagem de montagem e elaboração de programas simples. 
			\vspace{1em}
			\item Entender a estrutura e o funcionamento do barramento vislumbrando como é a troca de dados entre os diversos elementos constituintes do computador
			\vspace{1em}
			\item Conhecer da estrutura e funcionamento dos dispositivos básicos de entrada e saída, bem como das técnicas de comunicação entre eles, a memória e o processador. 

		\end{itemize}
	\end{frame}

	\begin{frame}[allowframebreaks]{Ementa}
		\begin{itemize}
			\item Computadores modernos. 
			\vspace{0.75em}
			\item Evolução das arquiteturas dos computadores.
			\vspace{0.75em}
			\item  Sistemas de numeração e aritmética binária. 
			\vspace{0.75em}
			\item Memória e representação de dados e instruções.
			\vspace{0.75em}
			\item  Processador, ciclo de instrução, formatos, endereçamento e programação em linguagem de montagem.
			\vspace{0.75em}
			\item  Dispositivos de entrada e saída. Sistemas de interconexão (barramentos).
			\vspace{0.75em}
			\item  Interfaceamento e técnicas de entrada e saída.
			\vspace{0.75em}
			\item  Hierarquia de memória. 
			\vspace{0.75em}
			\item Paralelismo a nível de instrução. 
			\vspace{0.75em}
			\item Arquiteturas paralelas.
		\end{itemize}
	\end{frame}
	
	\begin{frame}{Avaliações}
		\begin{eftable}
			\LARGE
			\begin{tabular}{c | c}
				\textcolor{white}{Avaliação} & 
				\textcolor{white}{Data} \\
				1${}^a$ V.A & 24/09/2020 \\
				2${}^a$ V.A & 12/11/2020 \\
				3${}^a$ V.A & 17/12/2020 \\			
			\end{tabular}
		\end{eftable}
	\end{frame}	
	
	\begin{frame}{Trabalhos – Regras gerais}
		\begin{itemize}
			\item Data de entrega - Avaliação
			\begin{itemize}
				\item Entregue na data correta – 100\%
				\item Entregue com até uma semana de atraso – 50\%
				\item Entregue com mais de uma semana de atraso – 0\%
			\end{itemize}
			\item Normas para os documentos
			\begin{itemize}
				\item \textbf{ABNT}
			\end{itemize}
			\item Plágio
			\begin{itemize}
				\item Em caso de plágio detectado em qualquer trabalho a nota será ZERO
			\end{itemize}
		\end{itemize}
	\end{frame}
	
	\begin{frame}{Pontuação}
		\begin{itemize}
			\item As notas para correção de provas e trabalhos serão de 0,25 em 0,25 pontos. Os arredondamentos só serão feitos após a nota fechada para lançamento no sistema. 

			\item Nas provas: 
			\begin{itemize}
				\item erros absurdos - 0 pontos; 
				\item raciocínio parcialmente correto: 25\%, 50\% ou 75\% da questão; 
				\item acertos integrais ou com erros irrelevantes: 100\%.

			\end{itemize}

		\end{itemize}
	\end{frame}
	
	\begin{frame}{Pontuação}
		\begin{center}
			\Huge \alert {Em caso de comprovação de cola em provas, trabalhos idênticos (inclusive com os mesmos erros) e plágios, a nota será \textbf{ZERADA}.}		
		\end{center}
	\end{frame}
	
	\begin{frame}{Modelo de prova}
		\begin{itemize}
			\item Questões de ENADE e concursos públicos
			\vspace{1em}
			\item Questões objetivas e discursivas
			\vspace{1em}
			\item Conteúdos de aulas de laboratório poderão ser cobrados na prova
		\end{itemize}
	\end{frame}
	
	\begin{frame}{Pontuação}
	
		\begin{itemize}
			\item Avaliação teórica - \textbf{0 a 50 pontos}.
			\vspace{1em}
			\item Avaliações processuais - \textbf{0 a 40 pontos} distribuídos da seguinte forma:
			\begin{itemize}
				\item Questionário Aula – 0 a 12 pontos
				\item APS – 0 a 5
				\item Outras Atividades – 0 a 33;
			\end{itemize}
		\end{itemize}
		
		Na 3VA os alunos matriculados em Projeto Interdisciplinar terão uma pontuação de 30 pontos referentes aos trabalhos desta disciplina. Estes pontos serão deduzidos da nota de \textit{Outras atividades}
	\end{frame}
	
	\begin{frame}{Material da disciplina}
		\begin{itemize}
			\item Disponibilizado no Lyceum
			\vspace{1em}
			\item Disponibilizado no AVA
			\vspace{1em}
			\item Disponível também no site \href{https://sites.google.com/site/professoralexandretannus}{\textcolor{blue}{https://sites.google.com/site/professoralexandretannus}}  
		\end{itemize}
	\end{frame}
	
	\begin{frame}{Bibliografia Básica}
		\begin{itemize}
			\item DELGADO, José. \textbf{Arquitetura de Computadores.} Editora LTC; 2ª 2010
			\vspace{1em}
			\item STALLINGS, William. \textbf{Arquitetura e organização de computadores : projeto para o desempenho.} 5. ed. São Paulo: Pearson  Education, 2002. 786 p.
			\vspace{1em}
			\item TANENBAUM, Andrew S.. \textbf{Introdução à organização de computadores.} Tradutor de Nery MACHADO FILHO. 4. ed.  Rio de Janeiro: LTC

		\end{itemize}
	\end{frame}
	
	\begin{frame}[allowframebreaks]{Bibliografia Complementar}
		\begin{itemize}
			\item HENNESSY, John L.; PATTERSON, David A. \textbf{Arquitetura de computadores: uma abordagem quantitativa.} Rio de  Janeiro: Campus, 2003. 827 p.
			\vspace{1em}
			\item MONTEIRO, Mário A.. \textbf{Introdução à organização de computadores.} 5. ed. Rio de Janeiro: LTC, 2012.
			\vspace{1em}
			\item PIVA JUNIOR, Dilermano. \textbf{Organização Básica de Computadores e Linguagem de Montagem.} Elsevier-Campus, 2012
			\framebreak
			\item PATTERSON, David A.; HENNESSY, John L.. \textbf{Organização e Projeto de Computadores}, 3.ed.  Campus-Elsevier.
			\vspace{1em}
			\item WEBER, Raul Fernando. \textbf{Fundamentos de arquitetura de computadores.} 3. ed. Sagra Luzzatto, 2004. 306 p.

		\end{itemize}
	\end{frame}
	
	\begin{frame}{}
		
	\end{frame}

\end{document}