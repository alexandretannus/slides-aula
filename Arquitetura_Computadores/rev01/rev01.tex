\documentclass[	aspectratio=169,
				xcolor=table]{beamer}


% Load general definitions
\usepackage[utf8]{inputenc}
%\usepackage[T1]{fontenc}
\usepackage[brazil]{babel}
\usepackage{amsmath}
\usepackage{amsfonts}
\usepackage{amssymb}
\usepackage{graphicx}
\usepackage{verbatim}
\usepackage{cancel}
\usepackage{askmaps}
\usepackage{tabularx}
\usepackage[table]{xcolor}
%\usepackage{tikz}
\usepackage{multirow}
\usepackage{mathtools}
\usepackage{color, colortbl}
\usepackage{etoolbox}
\usepackage{pbox}
\usepackage{changepage}
\usepackage{xpatch}
\usepackage{array}
\usepackage{marvosym}
\usepackage{tabu}
\usepackage{multicol}
\usepackage{listings}
\usepackage{underscore}
\usepackage{filecontents}
\usepackage[]{algorithm2e}
\usepackage{ragged2e}

\newcolumntype{P}[1]{>{\centering\arraybackslash}m{#1}}
\definecolor{Gray}{gray}{0.75}
\definecolor{Gray2}{gray}{0.85}

\definecolor{lightBlue}{HTML}{DAE8FC}
\definecolor{Blue}{RGB}{51, 51, 204}

%\useinnertheme[lily]{rounded}
\usetheme{UniEvangelica}
%\usetheme{Copenhagen}
%\usetheme{Berlin}
%\usecolortheme{dolphin}
\tolerance=1
\emergencystretch=\maxdimen
\hyphenpenalty=10000
\hbadness=10000

\setbeamertemplate{navigation symbols}{}%remove navigation symbols


\let\olditem=\item% 
\renewcommand{\item}{\olditem \justifying}%
\def\center{\trivlist \centering\item\relax}
\def\endcenter{\endtrivlist}

\setbeamertemplate{itemize/enumerate body begin}{\large}
\setbeamertemplate{itemize/enumerate subbody begin}{\large}

\setbeamertemplate{itemize item}{\raisebox{0.1ex}{$\blacktriangleright$}\hskip0.1em}
\setbeamertemplate{itemize subitem}{\raisebox{0.1ex}{$\blacktriangleright$}\hskip0.1em}

\newcommand{\greenarrow}{\textcolor{green}{\rotatebox[origin=c]{180}{\MVArrowDown}}}

\newcommand{\redarrow}{\textcolor{red}{\MVArrowDown}}

%\newcommand{\ftable}{
%	\begin{table}
%		\large
%		\centering
%		\rowcolors{1}{\ifnumless{\rownum}{2}{Blue}{lightBlue}}{}
%}

\newenvironment{eftable}{
	\begin{table}
		\large
		\centering
		\rowcolors{1}{}{Blue}
		\rowcolors{1}{\ifnumless{\rownum}{2}{Blue}{lightBlue}}{}
	}
	{
	\end{table}
}


%\setbeamertemplate{frametitle}
%{
%	%\vspace*{-2em}	
%	\insertframetitle
%
%	 %\textcolor{white}{\LARGE \insertframetitle}
%
%}

% Specific definitions
\title[]{Arquitetura de Computadores}
\subtitle[]{Revisão Prova 01}
\author[]{Prof. Alexandre Tannus}
\date{}

\begin{document}

	\begin{frame}
		\frametitle{Questão 1}
		\Large Resolva as seguintes operações. Escreva o resultado em formato hexadecimal
		
		\begin{itemize}
			\item $(57)_{8} + (D3)_{16}$
			\item $(10011001)_{2} + (58)_{10}$
			\item $(AB)_{16} + (712)_{8}$
			\item $(33)_{10} + (33)_{8} + (33)_{16}$	
			\item $(B4)_{16} - (90)_{10}$	
		\end{itemize}
		
	\end{frame}
	
	\begin{frame}
		\frametitle{Questão 2}
		A representação dos números e dos caracteres é importante para a organização dos computadores. Com relação às formas pelas quais os computadores podem armazenar esses números e caracteres, assinale a alternativa correta. 
		\begin{enumerate}[a]
			\large
			\item O byte é a unidade de informação mais básica nos sistemas de computação. 
			\item Um bit é um conjunto ordenado de 8 bytes. 
			\item Uma palavra pode ter tanto 32 bits quanto 64 bits, dependendo de seu tamanho.
			\item As duas bases mais utilizadas e mais importantes para a computação são a octal e a decimal. 
			\item Não há como realizar uma conversão de base, ou seja, um número que é representado na base decimal não pode ser convertido para a base binária.
		
		\end{enumerate}

	\end{frame}
	
	\begin{frame}
		\frametitle{Questão 3}
		O hardware da maior parte dos computadores atuais é baseado na implementação da arquitetura de Von Neumann. Um dos principais componentes da arquitetura mencionada é a Unidade Lógica e Aritmética (ULA). No contexto prático, dentre as alternativas abaixo, a ULA é responsável por executar a operação de:
		
		\begin{enumerate}[a]
			\large
			\item soma de dois números.
			\item controlar os periféricos de entrada e saída do sistema.
			\item armazenamento de dados na memória do computador.
			\item buscar, na memória, dados solicitados por um software.
			\item definir o próximo software a ser executado pelo processador		
		\end{enumerate}
	\end{frame}
	
	\begin{frame}
		\frametitle{Questão 4}
		Em relação à organização dos sistemas computacionais, as alternativas abaixo apresentam os principais componentes, EXCETO: 
		\begin{enumerate}[a]
			\item Componentes de E/S.
			\item Sistema operacional.
			\item Memória.
			\item Processador.
		\end{enumerate}
	\end{frame}
	
	\begin{frame}
		\frametitle{Questão 5}
		A Unidade Central de Processamento (UCP) é composta por um conjunto de componentes básico, EXCETO: 
		\begin{enumerate}[a]
			\item	Unidade de Controle.
			\item	Unidade de Entrada/Saída.
			\item	Unidade de Aritmética e Lógica. 
			\item	Conjunto de Registradores. 
			\item	Chipset.			
		\end{enumerate}
	\end{frame}

	\begin{frame}
		\frametitle{Questão 6}
		Um dos fatores que favoreceram o surgimento dos processadores de 64 bits foi a limitação do endereçamento de memória, pois processadores de 32 bits possuem a limitação de 4 GB de memória. Dessa forma, qual seria o limite, em GB, de um processador de 38 bits?
		\begin{enumerate}[a]
			\item	128
			\item	256
			\item	512
			\item	64
			\item	32  
		  	
		\end{enumerate}  
	\end{frame}
	
	\begin{frame}
		\frametitle{Questão 7}
		O acesso a dados em registradores internos da Unidade Central de Processamento (UCP): 
		\begin{enumerate}[a]
			\item produz uma cópia do dado em memória ROM; 
			\item não usa a memória RAM;
			\item pode acarretar perda de precisão;
			\item é tão rápido quanto o acesso a dados em memória RAM; 
			\item é mais lento que o acesso a dados em memória RAM.			
		\end{enumerate}
	\end{frame}
	
	\begin{frame}
		\frametitle{Questão 8}
		Em uma unidade central de processamento, o registrador cuja função precípua é indicar a próxima instrução a ser buscada para execução é denominado
		\begin{enumerate}[a]
			\item registrador de instrução (IR).
			\item acumulador (AX).
			\item registrador de base (BX).
			\item registrador de contadores (CX).
			\item contador de programa (PC).
		\end{enumerate}

	\end{frame}
	
	\begin{frame}
		\frametitle{Questão 9}
		Analise as afirmativas sobre a organização e arquitetura de computadores.
		\begin{enumerate}[i]
		\large
			\item A Unidade de Processamento Central (UCP) é composta por unidade lógica-aritmética, unidade de controle e registradores.
			\item Os registradores são memórias com pequena capacidade de armazenamento, porém com alto desempenho.
			\item A Unidade Lógica e Aritmética (ULA) é capaz de realizar as seguintes operações: adição, subtração, operações lógicas, e comparações. Mas esta unidade não é encarregada de realizar operações de deslocamento.
			\item Os registradores são encarregados de controlar as operações realizadas pela UCP, esta envia sinais que coordenam as operações internas.
			
		\end{enumerate}

	\end{frame}
	
	\begin{frame}
		\frametitle{Questão 10}
		Os registradores utilizados pela CPU e pela memória para comunicação e transferência de informações são, respectivamente:
		
		\begin{enumerate}[a]
			\item Contador de Instruções (CI) e Registrador de Dados de Memória (RDM).
			\item Registrador de Endereços de Memória (REM) e Contador de Instruções (CI).
			\item Registrador de Dados de Memória (RDM) e Registrador de Endereços de Memória (REM).
			\item  Decodificador de Instruções (DI) e Contador de Instruções (CI).
			\item  Decodificador de Instruções (DI) e Registrador de Dados de Memória (RDM).			
		\end{enumerate}
	\end{frame}
	
	\begin{frame}
		\frametitle{Questão 11}
		As seguintes características sobre registradores estão corretas, EXCETO: 
		
		\begin{enumerate}[a]
			\item Capacidade de armazenamento limitada.
			\item Alta velocidade.
			\item Residem fisicamente junto à CPU.
			\item Não volátil.			
			\item Principal objetivo é aumentar a eficiência de processamento
		\end{enumerate}
	\end{frame}
	
	\begin{frame}
		\frametitle{Questão 12}
		Computadores efetuam a subtração de números binários por meio de adição, na qual o número a ser subtraído é representado em complemento de um.
	\end{frame}
	
	\begin{frame}
		\frametitle{Questão 13}
		Dentro do conceito de organização de computadores, a UCP (Unidade Central de Processamento) desempenha um papel fundamental, sendo composta por diversas partes. Em particular, a Unidade de Controle é a parte da UCP responsável por
		\begin{enumerate}[a]
			\item armazenar resultados temporários.
			\item indicar a próxima instrução a ser buscada na memória, para execução.
			\item buscar instruções na memória principal e determinar o tipo dessas instruções.
			\item armazenar o código da instrução que está sendo correntemente executado.
			\item realizar operações como adição e subtração sobre os valores presentes nas suas entradas.

			
		\end{enumerate}
	\end{frame}

	\begin{frame}
		\frametitle{Questão 14}
		Instruções de máquina utilizam várias técnicas de endereçamento da memória.  Na técnica de endereçamento imediato, o
		\begin{enumerate}[a]
			\item valor do operando é especificado diretamente na instrução.
			\item endereço do operando é obtido diretamente do campo de endereço da instrução.
			\item endereço do operando é obtido diretamente do topo da pilha do sistema.
			\item endereço do operando encontra-se em um registrador predeterminado da CPU.
			\item campo de endereço da instrução contém um endereço de memória onde se encontra o endereço do operando.			
		\end{enumerate}
	\end{frame}
	
	\begin{frame}
		\frametitle{Questão 15}
		Considere:
		\begin{enumerate}[i]
			\item Acesso à memória limitado a instruções de carga e armazenamento (load e store). 
			\item Formato de instrução facilmente descodificável e de tamanho fixo. 
			\item Execução de instruções em um único ciclo de clock. 			
		\end{enumerate}


		I, II e III referem-se às características
		
		\begin{enumerate}[a]
			\item da BIOS.
			\item da EPROM.
			\item do programa montador.
			\item do processador RISC.
			\item do processador CISC.			
		\end{enumerate}
	
	\end{frame}

	\begin{frame}
		\frametitle{Questão 16}
		Computador com um Conjunto Reduzido de Instruções (RISC) é uma linha de arquitetura de processadores que favorece um conjunto simples e pequeno de instruções que levam aproximadamente a mesma quantidade de tempo para ser executadas. São
consideradas características típicas da organização RISC:
		\begin{enumerate}[a]
		\large
			\item oferecer suporte para linguagens de alto nível e facilitar o desenvolvimento de compiladores.
			\item prover o computador com um conjunto complexo de instruções e melhorar a execução de programas.
			\item manter poucos registradores e ter registradores especializados.
			\item otimizar o pipeline de instrução e apresentar um conjunto limitado de instruções com formato fixo.
			\item dispor grande conjunto de instruções e apresentar vários modos de endereçamento.				
		\end{enumerate}

	\end{frame}	

	\begin{frame}
		\frametitle{Questão 17}
		O processamento de instruções em um processador pode ser definido de forma simplificada nos seguintes passos: 
		\begin{enumerate}[a]
			\item Execução, Decodificação e Armazenamento. 
			\item Busca, Armazenamento e Execução. 
			\item Armazenamento, Busca e Decodificação.
			\item Decodificação, Busca e Armazenamento. 
			\item Busca, Decodificação e Execução.			
		\end{enumerate}
	\end{frame}
	
	\begin{frame}
		\frametitle{Questão 18}
		As plataformas de hardware de um computador estão intimamente relacionadas com o tipo de arquitetura adotada no processador por elas utilizado. A arquitetura Harvard, por exemplo, tem como característica marcante o fato de nela ocorrer a
		\begin{enumerate}[a]
			\large
			\item encriptação dos dados escritos na memória, para a umentar a segurança
			\item escrita dos dados em duas memórias distintas e redundantes, visando maior confiabilidade na recuperação de informações
			\item manipulação de dados na forma vetorial, agilizando o processamento de dados n-dimensionais
			\item utilização de um grande número de processadores trabalhando cooperativamente.
			\item separação de barramentos de comunicação para a memória de instruções de programa e para a memória de dados			
		\end{enumerate}
	\end{frame}
	
	\begin{frame}
		\frametitle{Questão 19}
		Um computador de 64 bits significa dizer que
		
		\begin{enumerate}[a]
			\item o clock oscila em frequência de 64 bits.
			\item os dados são armazenados na RAM em blocos de 64 bits.
			\item os dados são armazenados na cache em blocos de 64 bits.
			\item o tamanho da palavra manipulada pela UCP é de 64 bits.
			\item o tamanho do buffer para gravação no HD é de 64 bits.			
		\end{enumerate}


	\end{frame}


	
	\begin{frame}
		\frametitle{Respostas}
		
		\begin{eftable}
			\begin{tabular}{c c | c c | c c}
			\textcolor{white}{Questão} & \textcolor{white}{Resposta} & 
			\textcolor{white}{Questão} & \textcolor{white}{Resposta} & 
			\textcolor{white}{Questão} & \textcolor{white}{Resposta} \\
			2 & c & 3 & a & 4 & b \\ 
			5 & e & 6 & b & 7 & b \\
			8 & e & 9 & V-V-F-F & 10 & c \\			
			11 & d & 12 & F & 13 & c \\		
			14 & a & 15 & d & 16 & d \\		
			17 & e & 18 & e & 19 & d \\		
			\end{tabular}
		\end{eftable}
	\end{frame}
	


\end{document}
