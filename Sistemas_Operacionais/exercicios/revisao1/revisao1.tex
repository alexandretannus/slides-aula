\documentclass[aspectratio=169,
				xcolor=table]{beamer}


\setbeamertemplate{caption}{\raggedright\insertcaption\par}


% Load general definitions
\usepackage[utf8]{inputenc}
%\usepackage[T1]{fontenc}
\usepackage[brazil]{babel}
\usepackage{amsmath}
\usepackage{amsfonts}
\usepackage{amssymb}
\usepackage{graphicx}
\usepackage{verbatim}
\usepackage{cancel}
\usepackage{askmaps}
\usepackage{tabularx}
\usepackage[table]{xcolor}
%\usepackage{tikz}
\usepackage{multirow}
\usepackage{mathtools}
\usepackage{color, colortbl}
\usepackage{etoolbox}
\usepackage{pbox}
\usepackage{changepage}
\usepackage{xpatch}
\usepackage{array}
\usepackage{marvosym}
\usepackage{tabu}
\usepackage{multicol}
\usepackage{listings}
\usepackage{underscore}
\usepackage{filecontents}
\usepackage[]{algorithm2e}
\usepackage{ragged2e}

\newcolumntype{P}[1]{>{\centering\arraybackslash}m{#1}}
\definecolor{Gray}{gray}{0.75}
\definecolor{Gray2}{gray}{0.85}

\definecolor{lightBlue}{HTML}{DAE8FC}
\definecolor{Blue}{RGB}{51, 51, 204}

%\useinnertheme[lily]{rounded}
\usetheme{UniEvangelica}
%\usetheme{Copenhagen}
%\usetheme{Berlin}
%\usecolortheme{dolphin}
\tolerance=1
\emergencystretch=\maxdimen
\hyphenpenalty=10000
\hbadness=10000

\setbeamertemplate{navigation symbols}{}%remove navigation symbols


\let\olditem=\item% 
\renewcommand{\item}{\olditem \justifying}%
\def\center{\trivlist \centering\item\relax}
\def\endcenter{\endtrivlist}

\setbeamertemplate{itemize/enumerate body begin}{\large}
\setbeamertemplate{itemize/enumerate subbody begin}{\large}

\setbeamertemplate{itemize item}{\raisebox{0.1ex}{$\blacktriangleright$}\hskip0.1em}
\setbeamertemplate{itemize subitem}{\raisebox{0.1ex}{$\blacktriangleright$}\hskip0.1em}

\newcommand{\greenarrow}{\textcolor{green}{\rotatebox[origin=c]{180}{\MVArrowDown}}}

\newcommand{\redarrow}{\textcolor{red}{\MVArrowDown}}

%\newcommand{\ftable}{
%	\begin{table}
%		\large
%		\centering
%		\rowcolors{1}{\ifnumless{\rownum}{2}{Blue}{lightBlue}}{}
%}

\newenvironment{eftable}{
	\begin{table}
		\large
		\centering
		\rowcolors{1}{}{Blue}
		\rowcolors{1}{\ifnumless{\rownum}{2}{Blue}{lightBlue}}{}
	}
	{
	\end{table}
}


%\setbeamertemplate{frametitle}
%{
%	%\vspace*{-2em}	
%	\insertframetitle
%
%	 %\textcolor{white}{\LARGE \insertframetitle}
%
%}

% Specific definitions
\institute[]{\uppercase{Engenharia de Software}}
\title[]{Sistemas Operacionais}
\subtitle[]{\uppercase{Exercícios de Revisão}}
\author[]{Prof. M.e Alexandre Tannus}
\date{Anápolis - 2019.2}

%\AtBeginSection{\frame{\tableofcontents[currentsection]}}

\begin{document}
	\begin{frame}
		\titlepage		
	\end{frame}

	\begin{frame}{TRF - 2ª REGIÃO 2017}
		“Sistemas Operacionais gerenciam aplicações e outras abstrações de software, como máquinas virtuais. Dessa forma, as finalidades primárias de um sistema operacional são __________ aplicações a interagir com um hardware de computador e __________ os recursos de hardware e software de um sistema.” Assinale a alternativa que completa correta e sequencialmente a afirmativa anterior.
		
		\begin{enumerate}[a]		
			\item gerenciar / habilitar 
			\item habilitar / gerenciar
			\item habilitar / suspender 
			\item hospedar / amplificar
		\end{enumerate}
	\end{frame}		

	\begin{frame}{TRF - 2ª REGIÃO 2017}
		“Sistemas Operacionais gerenciam aplicações e outras abstrações de software, como máquinas virtuais. Dessa forma, as finalidades primárias de um sistema operacional são __________ aplicações a interagir com um hardware de computador e __________ os recursos de hardware e software de um sistema.” Assinale a alternativa que completa correta e sequencialmente a afirmativa anterior.
		
		\begin{enumerate}[a]		
			\item gerenciar / habilitar 
			\item \alert{habilitar / gerenciar}
			\item habilitar / suspender 
			\item hospedar / amplificar
		\end{enumerate}
	\end{frame}		

		
	\begin{frame}{Petrobras 2012}
		O mecanismo pelo qual programas dos usuários solicitam serviços ao núcleo do sistema operacional é denominado
		\begin{enumerate}[a]
			\item biblioteca do sistema
			\item chamada do sistema
			\item editor de ligação 
			\item shell de comandos
			\item ligação dinâmica
		\end{enumerate}
	\end{frame}

	\begin{frame}{Petrobras 2012}
		O mecanismo pelo qual programas dos usuários solicitam serviços ao núcleo do sistema operacional é denominado
		\begin{enumerate}[a]
			\item biblioteca do sistema
			\item \alert{chamada do sistema}
			\item editor de ligação 
			\item shell de comandos
			\item ligação dinâmica
		\end{enumerate}
	\end{frame}

	\begin{frame}{CEFET-MG 2014}
		
		São componentes internos ao Sistema Operacional, os princípios de
		\begin{enumerate}[a]
			\item sistema de interpretação de comandos, barramento, memória, discos e sistema de processamento.

			\item gerência de processos, shell, gerenciamento de diretórios, memória virtual e linguagem de programação.

			\item sistema cliente-servidor, máquinas virtuais, linguagem de máquina, sistema multitarefa e multiprocessamento.

			\item gerência de memória, controle de processos, controle de E/S, sistema de arquivos e sistema de segurança.

			\item ambiente operacional, sistema de aplicativos, browser, controle de leitura/gravação e sistema de armazenamento.

		\end{enumerate}
	\end{frame}


	\begin{frame}{CEFET-MG 2014}
		
		São componentes internos ao Sistema Operacional, os princípios de
		\begin{enumerate}[a]
			\item sistema de interpretação de comandos, barramento, memória, discos e sistema de processamento.

			\item gerência de processos, shell, gerenciamento de diretórios, memória virtual e linguagem de programação.

			\item sistema cliente-servidor, máquinas virtuais, linguagem de máquina, sistema multitarefa e multiprocessamento.

			\item \alert{gerência de memória, controle de processos, controle de E/S, sistema de arquivos e sistema de segurança.}

			\item ambiente operacional, sistema de aplicativos, browser, controle de leitura/gravação e sistema de armazenamento.

		\end{enumerate}
	\end{frame}

	\begin{frame}{UFJF 2017}
		Sobre sistemas operacionais em geral, é INCORRETO afirmar que:
		\begin{enumerate}[a]
			\small
			\item Um sistema operacional é responsável pelo gerenciamento dos recursos de hardware de um computador, permitindo o uso destes recursos por programas em execução.  

			\item Um sistema operacional oferece uma interface ao usuário que, no mínimo, permite a escolha e execução de programas. 

			\item Um sistema operacional normalmente suporta um ou mais tipos de sistemas de arquivos, de forma a permitir o armazenamento da informação pelos usuários ao manipularem seus programas.  

			\item Sistemas operacionais multitarefa são construídos especificamente para computadores com mais de uma CPU, para oferecer o suporte adequado à execução de tarefas concorrentemente. 

			\item Sistemas operacionais multiusuário devem ser multitarefa, para oferecer o suporte adequado a vários usuários concorrentemente.

		\end{enumerate}

	\end{frame}
	

	\begin{frame}{UFJF 2017}
		Sobre sistemas operacionais em geral, é INCORRETO afirmar que:
		\begin{enumerate}[a]
			\small
			\item Um sistema operacional é responsável pelo gerenciamento dos recursos de hardware de um computador, permitindo o uso destes recursos por programas em execução.  

			\item Um sistema operacional oferece uma interface ao usuário que, no mínimo, permite a escolha e execução de programas. 

			\item Um sistema operacional normalmente suporta um ou mais tipos de sistemas de arquivos, de forma a permitir o armazenamento da informação pelos usuários ao manipularem seus programas.  

			\item \alert{Sistemas operacionais multitarefa são construídos especificamente para computadores com mais de uma CPU, para oferecer o suporte adequado à execução de tarefas concorrentemente. }

			\item Sistemas operacionais multiusuário devem ser multitarefa, para oferecer o suporte adequado a vários usuários concorrentemente.

		\end{enumerate}

	\end{frame}	
	
	\begin{frame}{CEGÁS 2017 }
		Assinale a alternativa correta para o conceito de sistemas operacionais. 
		\begin{enumerate}[a]
			
			\small
			\item Trata-se de um conjunto de programas que tem como funções básicas administrar processos e pessoas que estejam operando o computador, a memória principal, armazenamento secundária bem como o sistema de entrada e de saída E/S. Seus componentes básicos são administração de arquivos, sistemas de proteção, comunicação e interpretador de comandos do sistema. 
			\item Trata-se de um conjunto de programas que tem como funções básicas administrar processos, a memória principal, armazenamento secundária bem como o sistema de entrada e de saída E/S. Seus componentes básicos são administração de arquivos, sistemas de proteção, comunicação e interpretador de comandos do sistema. 
			\item Trata-se de um conjunto de programas que tem como funções básicas administrar processos, a memória principal, armazenamento secundária bem como o sistema de entrada e de saída E/S. Seus componentes básicos são administração de arquivos, sistemas de proteção, comunicação e interpretador de comandos do sistema. É também responsável pela criptografia das informações para garantir a segurança digital. 
			\item Trata-se de um conjunto de programas que tem como funções básicas gerenciamento de banco de dados, administrar processos, a memória principal, armazenamento secundária bem como o sistema de entrada e de saída E/S. Seus componentes básicos são administração de arquivos, sistemas de proteção, comunicação e interpretador de comandos do sistema.

		\end{enumerate}
	\end{frame}

	\begin{frame}{CEGÁS 2017 }
		Assinale a alternativa correta para o conceito de sistemas operacionais. 
		\begin{enumerate}[a]			
			\small
			\item Trata-se de um conjunto de programas que tem como funções básicas administrar processos e pessoas que estejam operando o computador, a memória principal, armazenamento secundária bem como o sistema de entrada e de saída E/S. Seus componentes básicos são administração de arquivos, sistemas de proteção, comunicação e interpretador de comandos do sistema. 
			\item \alert{Trata-se de um conjunto de programas que tem como funções básicas administrar processos, a memória principal, armazenamento secundária bem como o sistema de entrada e de saída E/S. Seus componentes básicos são administração de arquivos, sistemas de proteção, comunicação e interpretador de comandos do sistema.} 
			\item Trata-se de um conjunto de programas que tem como funções básicas administrar processos, a memória principal, armazenamento secundária bem como o sistema de entrada e de saída E/S. Seus componentes básicos são administração de arquivos, sistemas de proteção, comunicação e interpretador de comandos do sistema. É também responsável pela criptografia das informações para garantir a segurança digital. 
			\item Trata-se de um conjunto de programas que tem como funções básicas gerenciamento de banco de dados, administrar processos, a memória principal, armazenamento secundária bem como o sistema de entrada e de saída E/S. Seus componentes básicos são administração de arquivos, sistemas de proteção, comunicação e interpretador de comandos do sistema.
		\end{enumerate}
	\end{frame}


	\begin{frame}{UFES 2016}
		Um Sistema Operacional funciona como uma interface entre um programa de usuário e o hardware e oferece uma variedade de serviços e funções de supervisão. NÃO é uma das tarefas clássicas de um Sistema Operacional 
		\begin{enumerate}[a]
			\item o controle da frequência de clock para permitir execuções mais rápidas usando overclocking. 
			\item o tratamento das operações básicas de entrada e saída das aplicações. 
			\item a proteção dos recursos compartilhados entre as múltiplas aplicações em execução numa máquina. 
			\item a alocação de memória para as aplicações. 
			\item a criação e destruição de processos associados às aplicações.			
		\end{enumerate}
	\end{frame}
	
	\begin{frame}{UFES 2016}
		Um Sistema Operacional funciona como uma interface entre um programa de usuário e o hardware e oferece uma variedade de serviços e funções de supervisão. NÃO é uma das tarefas clássicas de um Sistema Operacional 
		\begin{enumerate}[a]
			\item \alert{o controle da frequência de clock para permitir execuções mais rápidas usando overclocking. }
			\item o tratamento das operações básicas de entrada e saída das aplicações. 
			\item a proteção dos recursos compartilhados entre as múltiplas aplicações em execução numa máquina. 
			\item a alocação de memória para as aplicações. 
			\item a criação e destruição de processos associados às aplicações.			
		\end{enumerate}
	\end{frame}
	
	
	\begin{frame}{ENADE 2017}
		“Conceitualmente, cada processo tem sua própria CPU virtual. É claro que, na realidade, a CPU troca a execução, a todo momento, de um processo para outro, mas, para entender esse sistema, é muito mais fácil pensar em um conjunto de processos sendo executados (pseudo) paralelamente do que tentar controlar o modo como a CPU faz esses chaveamentos” – TANENBAUM, 2010
	
		De acordo com o exposto, o conceito descrito denomina-se:
		\begin{enumerate}[a]
			\item Thread
			\item Multiprocessador 
			\item Multiprogramação 
			\item Processo monothread
			\item Máquina de estados finitos			
		\end{enumerate}
	\end{frame}
	
	\begin{frame}{ENADE 2017}
		“Conceitualmente, cada processo tem sua própria CPU virtual. É claro que, na realidade, a CPU troca a execução, a todo momento, de um processo para outro, mas, para entender esse sistema, é muito mais fácil pensar em um conjunto de processos sendo executados (pseudo) paralelamente do que tentar controlar o modo como a CPU faz esses chaveamentos” – TANENBAUM, 2010
	
		De acordo com o exposto, o conceito descrito denomina-se:
		\begin{enumerate}[a]
			\item Thread
			\item Multiprocessador 
			\item \alert{Multiprogramação }
			\item Processo monothread
			\item Máquina de estados finitos			
		\end{enumerate}
	\end{frame}

	\begin{frame}{ITAIPU BINACIONAL 2017}
		Troca de contexto é uma tarefa efetuada pelo Sistema Operacional na gerência de tarefas. A troca de contexto consiste em:
		\begin{enumerate}[a]
			\item trocar o usuário logado no Sistema Operacional, para que outro usuário possa utilizá-lo sem interferência nas informações do usuário anterior.
			\item interromper a execução de aplicativos críticos.
			\item salvar informações de uma tarefa para que o processador possa ser entregue a outra, carregando seu contexto.
			\item recarregar o contexto do usuário para restaurar o estado da máquina.
			\item trocar a tarefa que gerencia as impressoras instaladas na máquina. 			
		\end{enumerate}
	\end{frame}

	\begin{frame}{ITAIPU BINACIONAL 2017}
		Troca de contexto é uma tarefa efetuada pelo Sistema Operacional na gerência de tarefas. A troca de contexto consiste em:
		\begin{enumerate}[a]
			\item trocar o usuário logado no Sistema Operacional, para que outro usuário possa utilizá-lo sem interferência nas informações do usuário anterior.
			\item interromper a execução de aplicativos críticos.
			\item \alert{salvar informações de uma tarefa para que o processador possa ser entregue a outra, carregando seu contexto.}
			\item recarregar o contexto do usuário para restaurar o estado da máquina.
			\item trocar a tarefa que gerencia as impressoras instaladas na máquina. 			
		\end{enumerate}
	\end{frame}

	\begin{frame}{}
	Em um sistema operacional típico, os estados de um processo são:
		\begin{enumerate}[a]
			\item Ativo, Desocupado, Finalizado e Executando.
			\item Bloqueado, Desbloqueado, Ativo e Suspenso.
			\item Executando, Bloqueado e Pronto.
			\item Parado, Ocupado, em Execução e Finalizado.
			\item Pronto, Terminado, Ativo e Processando.
			
		\end{enumerate}
	\end{frame}

	\begin{frame}{}
	Em um sistema operacional típico, os estados de um processo são:
		\begin{enumerate}[a]
			\item Ativo, Desocupado, Finalizado e Executando.
			\item Bloqueado, Desbloqueado, Ativo e Suspenso.
			\item \alert{Executando, Bloqueado e Pronto.}
			\item Parado, Ocupado, em Execução e Finalizado.
			\item Pronto, Terminado, Ativo e Processando.			
		\end{enumerate}
	\end{frame}


	\begin{frame}{UFPE 2010}
		Sobre gerência de processamento, assinale a alternativa incorreta.
		\begin{enumerate}[a]
			\item Uma política de escalonamento é composta por critérios estabelecidos para determinar qual processo em estado de pronto será escolhido para fazer uso do processador
			\item O escalonador é uma rotina do sistema operacional que tem como principal função implementar os critérios da política de escalonamento.
			\item Tempo de processador ou tempo de UCP é o tempo que um processo leva no estado de execução durante seu processamento.
			\item No escalonamento preemptivo, o sistema operacional pode interromper um processo em execução e passá-lo para o estado de pronto, com o objetivo de alocar outro processo na UCP.
			\item Preempção por prioridade, ocorre quando o sistema operacional interrompe o processo em execução em função da expiração da sua fatia de tempo, substituindo-o por outro processo.			
		\end{enumerate}
	\end{frame}
	
	\begin{frame}{UFPE 2010}
		Sobre gerência de processamento, assinale a alternativa incorreta.
		\begin{enumerate}[a]
			\item Uma política de escalonamento é composta por critérios estabelecidos para determinar qual processo em estado de pronto será escolhido para fazer uso do processador
			\item O escalonador é uma rotina do sistema operacional que tem como principal função implementar os critérios da política de escalonamento.
			\item Tempo de processador ou tempo de UCP é o tempo que um processo leva no estado de execução durante seu processamento.
			\item No escalonamento preemptivo, o sistema operacional pode interromper um processo em execução e passá-lo para o estado de pronto, com o objetivo de alocar outro processo na UCP.
			\item \alert{Preempção por prioridade, ocorre quando o sistema operacional interrompe o processo em execução em função da expiração da sua fatia de tempo, substituindo-o por outro processo.}
		\end{enumerate}
	\end{frame}

	\begin{frame}{IF Baiano 2017}
		Em um sistema operacional, frequentemente um processo precisa interagir com outro processo, ainda que cada processo seja uma entidade independente. Além disso, em um ambiente multiprogramado, um processo não ocupa todo o tempo do processador. Por conta desses fatores, um processo pode estar nos seguintes estados: Bloqueado, Em execução e Pronto.
	São transições válidas de estados entre processos, EXCETO
		\begin{enumerate}[a]
			\item Bloqueado $\to$ Pronto
			\item Pronto $\to$ Bloqueado 
			\item Em execução $\to$ Pronto 
			\item Pronto $\to$ Em execução
			\item Em execução $\to$ Bloqueado			
		\end{enumerate}
	\end{frame}

	\begin{frame}{IF Baiano 2017}
		Em um sistema operacional, frequentemente um processo precisa interagir com outro processo, ainda que cada processo seja uma entidade independente. Além disso, em um ambiente multiprogramado, um processo não ocupa todo o tempo do processador. Por conta desses fatores, um processo pode estar nos seguintes estados: Bloqueado, Em execução e Pronto.
	São transições válidas de estados entre processos, EXCETO
		\begin{enumerate}[a]
			\item Bloqueado $\to$ Pronto
			\item \alert{Pronto $\to$ Bloqueado }
			\item Em execução $\to$ Pronto 
			\item Pronto $\to$ Em execução
			\item Em execução $\to$ Bloqueado			
		\end{enumerate}
	\end{frame}
	
	\begin{frame}{Prefeitura de Caieiras 2015}
		Em um sistema operacional típico, vários processos podem se encontrar no estado “pronto” em um dado instante. A gerência do processador efetua a escolha de qual desses processos receberá o processador. Essa escolha atende a critérios previamente definidos, que fazem parte da política de
		\begin{enumerate}[a]
			\item alocação de memória.
			\item escalonamento de processos.
			\item minimização do throughput do sistema
			\item particionamento da Unidade Central de Processamento.
			\item virtualização da memória principal.		
		\end{enumerate}
	\end{frame}	
	
	\begin{frame}{Prefeitura de Caieiras 2015}
		Em um sistema operacional típico, vários processos podem se encontrar no estado “pronto” em um dado instante. A gerência do processador efetua a escolha de qual desses processos receberá o processador. Essa escolha atende a critérios previamente definidos, que fazem parte da política de
		\begin{enumerate}[a]
			\item alocação de memória.
			\item \alert{escalonamento de processos.}
			\item minimização do throughput do sistema
			\item particionamento da Unidade Central de Processamento.
			\item virtualização da memória principal.		
		\end{enumerate}
	\end{frame}	
	
	\begin{frame}{DPE-RS 2013}
		Um dos aspectos mais importantes dos sistemas operacionais é a capacidade de realizar multiprogramação. Sobre este assunto, é INCORRETO afirmar:
		\begin{enumerate}[a]
			\small
			\item A multiprogramação aumenta a utilização da CPU organizando os jobs (código e dados) prontos para serem executados, de modo que a CPU tenha sempre um deles para executar, não ficando ociosa.
			\item Em sistemas de tempo compartilhado, a CPU executa múltiplos jobs alternando-se entre eles, mas as mudanças ocorrem com tanta frequência que os usuários ficam impedidos de interagir com os programas enquanto estão sendo executados.
			\item O SO mantém vários jobs na memória simultaneamente; como a memória costuma ser pequena para acomodar todos os jobs, estes são mantidos inicialmente em disco na fila de jobs, que é composta por jobs que aguardam alocação na memória principal.
			\item A multiprogramação pode ser comparada ao trabalho de um advogado: ele trabalha para vários clientes; enquanto um caso está aguardando julgamento ou esperando documentos, ele pode trabalhar em outro caso.
			\item O tempo compartilhado (ou multitarefa) é uma extensão lógica da multiprogramação. Apenas um pequeno tempo de CPU é dado a cada usuário, de forma que ele tem a impressão de que todo o sistema de computação está dedicado exclusivamente ao seu programa.
			
		\end{enumerate}

	\end{frame}	
	
	\begin{frame}{DPE-RS 2013}
		Um dos aspectos mais importantes dos sistemas operacionais é a capacidade de realizar multiprogramação. Sobre este assunto, é INCORRETO afirmar:
		\begin{enumerate}[a]
			\small
			\item A multiprogramação aumenta a utilização da CPU organizando os jobs (código e dados) prontos para serem executados, de modo que a CPU tenha sempre um deles para executar, não ficando ociosa.
			\item \alert{Em sistemas de tempo compartilhado, a CPU executa múltiplos jobs alternando-se entre eles, mas as mudanças ocorrem com tanta frequência que os usuários ficam impedidos de interagir com os programas enquanto estão sendo executados.}
			\item O SO mantém vários jobs na memória simultaneamente; como a memória costuma ser pequena para acomodar todos os jobs, estes são mantidos inicialmente em disco na fila de jobs, que é composta por jobs que aguardam alocação na memória principal.
			\item A multiprogramação pode ser comparada ao trabalho de um advogado: ele trabalha para vários clientes; enquanto um caso está aguardando julgamento ou esperando documentos, ele pode trabalhar em outro caso.
			\item O tempo compartilhado (ou multitarefa) é uma extensão lógica da multiprogramação. Apenas um pequeno tempo de CPU é dado a cada usuário, de forma que ele tem a impressão de que todo o sistema de computação está dedicado exclusivamente ao seu programa.
			
		\end{enumerate}

	\end{frame}	
	
	\begin{frame}{PROCEMPA 2012}
		Um sistema operacional que permite multiprogramação está rodando em uma máquina que possui um único processador. Nesse sistema operacional, ocorreu que um processo foi levado do estado de EXECUTANDO (running) para o estado de APTO (ready). Considerando que o escalonador desse sistema operacional é não preemptivo, assinale dentre as alternativas abaixo aquela que corresponde ao evento que gerou essa transição de estados.
		\begin{enumerate}[a]
			\item Interrupção do processador pelo circuito do disco, indicando a disponibilidade de um bloco de dados.
			\item Seleção, pelo escalonador, de outro processo para entrar em execução.
			\item Chamada de sistema efetuada pelo processo para o envio de um pacote de dados pela rede.
			\item Liberação voluntária do processador pelo processo.
			\item Sinalização para o processo de que um periférico terminou uma operação solicitada.
			
		\end{enumerate}
	\end{frame}	
	
	\begin{frame}{PROCEMPA 2012}
		Um sistema operacional que permite multiprogramação está rodando em uma máquina que possui um único processador. Nesse sistema operacional, ocorreu que um processo foi levado do estado de EXECUTANDO (running) para o estado de APTO (ready). Considerando que o escalonador desse sistema operacional é não preemptivo, assinale dentre as alternativas abaixo aquela que corresponde ao evento que gerou essa transição de estados.
		\begin{enumerate}[a]
			\item Interrupção do processador pelo circuito do disco, indicando a disponibilidade de um bloco de dados.
			\item Seleção, pelo escalonador, de outro processo para entrar em execução.
			\item Chamada de sistema efetuada pelo processo para o envio de um pacote de dados pela rede.
			\item \alert{Liberação voluntária do processador pelo processo.}
			\item Sinalização para o processo de que um periférico terminou uma operação solicitada.
			
		\end{enumerate}
	\end{frame}	
	
	
	
	\begin{frame}{}
	\end{frame}	
\end{document}
