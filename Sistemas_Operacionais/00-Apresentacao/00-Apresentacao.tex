\documentclass[aspectratio=169,
				xcolor=table]{beamer}

% Load general definitions
\usepackage[utf8]{inputenc}
%\usepackage[T1]{fontenc}
\usepackage[brazil]{babel}
\usepackage{amsmath}
\usepackage{amsfonts}
\usepackage{amssymb}
\usepackage{graphicx}
\usepackage{verbatim}
\usepackage{cancel}
\usepackage{askmaps}
\usepackage{tabularx}
\usepackage[table]{xcolor}
%\usepackage{tikz}
\usepackage{multirow}
\usepackage{mathtools}
\usepackage{color, colortbl}
\usepackage{etoolbox}
\usepackage{pbox}
\usepackage{changepage}
\usepackage{xpatch}
\usepackage{array}
\usepackage{marvosym}
\usepackage{tabu}
\usepackage{multicol}
\usepackage{listings}
\usepackage{underscore}
\usepackage{filecontents}
\usepackage[]{algorithm2e}
\usepackage{ragged2e}

\newcolumntype{P}[1]{>{\centering\arraybackslash}m{#1}}
\definecolor{Gray}{gray}{0.75}
\definecolor{Gray2}{gray}{0.85}

\definecolor{lightBlue}{HTML}{DAE8FC}
\definecolor{Blue}{RGB}{51, 51, 204}

%\useinnertheme[lily]{rounded}
\usetheme{UniEvangelica}
%\usetheme{Copenhagen}
%\usetheme{Berlin}
%\usecolortheme{dolphin}
\tolerance=1
\emergencystretch=\maxdimen
\hyphenpenalty=10000
\hbadness=10000

\setbeamertemplate{navigation symbols}{}%remove navigation symbols


\let\olditem=\item% 
\renewcommand{\item}{\olditem \justifying}%
\def\center{\trivlist \centering\item\relax}
\def\endcenter{\endtrivlist}

\setbeamertemplate{itemize/enumerate body begin}{\large}
\setbeamertemplate{itemize/enumerate subbody begin}{\large}

\setbeamertemplate{itemize item}{\raisebox{0.1ex}{$\blacktriangleright$}\hskip0.1em}
\setbeamertemplate{itemize subitem}{\raisebox{0.1ex}{$\blacktriangleright$}\hskip0.1em}

\newcommand{\greenarrow}{\textcolor{green}{\rotatebox[origin=c]{180}{\MVArrowDown}}}

\newcommand{\redarrow}{\textcolor{red}{\MVArrowDown}}

%\newcommand{\ftable}{
%	\begin{table}
%		\large
%		\centering
%		\rowcolors{1}{\ifnumless{\rownum}{2}{Blue}{lightBlue}}{}
%}

\newenvironment{eftable}{
	\begin{table}
		\large
		\centering
		\rowcolors{1}{}{Blue}
		\rowcolors{1}{\ifnumless{\rownum}{2}{Blue}{lightBlue}}{}
	}
	{
	\end{table}
}


%\setbeamertemplate{frametitle}
%{
%	%\vspace*{-2em}	
%	\insertframetitle
%
%	 %\textcolor{white}{\LARGE \insertframetitle}
%
%}

% Specific definitions
\institute[]{\uppercase{Engenharia de Software}}
\title[]{Sistemas Operacionais}
\subtitle[]{\uppercase{Conceitos Básicos}}
\author[]{Prof. M.e Alexandre Tannus}
\date{Anápolis - 2021.1}

%\AtBeginSection{\frame{\tableofcontents[currentsection]}}

\begin{document}
	\begin{frame}
		\titlepage
	\end{frame}
	
	\begin{frame}
		\frametitle{Professor Alexandre Tannus}
		\begin{center}
			\centering
			\huge Bacharel em Engenharia da Computação
			\begin{figure}
				\centering
				\includegraphics[height=2cm, keepaspectratio]{../figs/puc.jpg} 			
			\end{figure}
			\huge Mestre em Engenharia Elétrica
			\begin{figure}
			\centering
				\includegraphics[height=2cm, keepaspectratio]{../figs/ufmg} 		
			\end{figure}
		\end{center}
	\end{frame}
	
	\begin{frame}
		\frametitle{Objetivos}
		\begin{itemize}
			\item Compreender a função de um sistema operacional.
			\vspace{1em}
			\item Compreender o quanto os sistema operacionais influenciam na execução de aplicações. 
			\vspace{1em}
			\item Entender como os sistemas operacionais gerenciam os recursos de: processamento, memória, comunicação, entrada/saída, sistema de arquivos.
		\end{itemize}
	\end{frame}



	\begin{frame}
		\frametitle{Ementa}
		\begin{itemize}
			\item Estrutura/organização de Sistemas Operacionais 
			\vspace{1em}
			\item Processos e Threads
			\vspace{1em}
			\item Deadlocks
			\vspace{1em}
			\item Gerenciamento de Memória
			\vspace{1em}
			\item Entrada/Saída e sistemas de arquivos
		\end{itemize}
	\end{frame}

	\begin{frame}
		\frametitle{Avaliações}
		\begin{eftable}
			\begin{tabular}{c | c}
				\textcolor{white}{Avaliação} & 
				\textcolor{white}{Data} \\
				1 ${}^a$ V.A & 06/04/2021 \\
				2 ${}^a$ V.A & 11/05/2021 \\
				3 ${}^a$ V.A & 15/06/2021 \\			
			\end{tabular}
		\end{eftable}
	\end{frame}	
	
	\begin{frame}
		\frametitle{Trabalhos – Regras gerais}
		\begin{itemize}
			\item Data de entrega - Avaliação
			\begin{itemize}
				\item Entregue na data correta – 100\%
				\item Entregue com até uma semana de atraso – 50\%
				\item Entregue com mais de uma semana de atraso – 0\%
			\end{itemize}
			\item Normas para os documentos
			\begin{itemize}
				\item \textbf{ABNT}
			\end{itemize}
			\item Plágio
			\begin{itemize}
				\item Em caso de plágio detectado em qualquer trabalho a nota será ZERO
			\end{itemize}
		\end{itemize}
	\end{frame}
	
	\begin{frame}
		\frametitle{Pontuação}
		\begin{itemize}
			\item As notas para correção de provas e trabalhos serão de 0,25 em 0,25 pontos. Os arredondamentos só serão feitos após a nota fechada para lançamento no sistema. 

			\item Nas provas: 
			\begin{itemize}
				\item erros absurdos - 0 pontos; 
				\item raciocínio parcialmente correto: 25\%, 50\% ou 75\% da questão; 
				\item acertos integrais ou com erros irrelevantes: 100\% .

			\end{itemize}

		\end{itemize}
	\end{frame}
	
	\begin{frame}
		\frametitle{Pontuação}
		\begin{center}
			\Huge \alert {Em caso de comprovação de cola em provas, trabalhos idênticos (inclusive com os mesmos erros) e plágios, a nota será \textbf{ZERADA}.}		
		\end{center}
	\end{frame}
	
	\begin{frame}
		\frametitle{Modelo de prova}
		\begin{itemize}
			\item Questões de ENADE e concursos públicos
			\vspace{1em}
			\item Questões objetivas e discursivas
			\vspace{1em}
			\item Conteúdos de aulas de laboratório poderão ser cobrados na prova
		\end{itemize}
	\end{frame}
	
	\begin{frame}
		\frametitle{Material da disciplina}
		\begin{itemize}
			\item Disponibilizado no Lyceum
			\vspace{1em}
			\item Disponível também no site \href{https://sites.google.com/site/professoralexandretannus}{\textcolor{blue}{https://sites.google.com/site/professoralexandretannus}}  
		\end{itemize}
	\end{frame}
	
	\begin{frame}
		\frametitle{Bibliografia Básica}
		\begin{itemize}
			\vspace{1em}
			\item SILBERSCHATZ, A.; GALVIN, P. B.; GAGNE, G.. \textbf{Fundamentos de sistemas operacionais: princípios básicos.} Rio de Janeiro: LTC – Livros Técnicos e Científicos, 2013.
			\vspace{1em}
			\item COULOURIS, G.; DOLLIMORE, J.; KINDBERG, T. \textbf{Sistemas distribuídos: conceitos e projeto.} 5. ed. Porto Alegre, RS, Brasil: Bookman, 2013. 1048 p. 
			\vspace{1em}
			\item DEITEL, H. M. e outros. \textbf{Sistemas Operacionais.} 3ª edição, São Paulo: Pearson Prentice Hall, 2005.

		\end{itemize}
	\end{frame}
	
	\begin{frame}
		\frametitle{Bibliográfia Complementar}
		\begin{itemize}
			\item TANEMBAUM, A. S. Sistemas Operacionais Modernos, São Paulo: Prentice-Hall, 2003.
			\vspace{1em}
			\item SILBERSCHATZ, A.; GALVIN, P.B.; GAGNE, G. \textbf{Sistemas operacionais com Java.} 7. ed. Rio de Janeiro, RJ, Brasil: Elsevier, 2008. 673 p.
			\item CORTÊS, Pedro Luiz. \textbf{Sistemas operacionais: fundamentos.} 2. ed. São Paulo: São Paulo, 2005.

		\end{itemize}
	\end{frame}

	\begin{frame}
		\frametitle{Bibliográfia Complementar}
		\begin{itemize}
		
			\item OLIVEIRA, R.S.; CARISSIMI, A.S.; TOSCANI, S.S. \textbf{Sistemas operacionais e programação concorrente.} Sagra Luzzatto, 2003.
			\vspace{1em}
			\item TOSCANI, S.S.; OLIVEIRA, R.S.; CARISSIMI, A.S. \textbf{Sistemas operacionais e programação concorrente.} Sagra Luzzatto, 2003.
			\vspace{1em}
			\item FLYNN, I.M. \textbf{Introdução aos sistemas operacionais.} Thonson, 2002.

		\end{itemize}
	\end{frame}
	
	\begin{frame}
	
	\end{frame}
\end{document}