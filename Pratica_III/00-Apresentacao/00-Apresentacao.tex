\documentclass[aspectratio=169,
				xcolor=table]{beamer}

% Load general definitions
\usepackage[utf8]{inputenc}
%\usepackage[T1]{fontenc}
\usepackage[brazil]{babel}
\usepackage{amsmath}
\usepackage{amsfonts}
\usepackage{amssymb}
\usepackage{graphicx}
\usepackage{verbatim}
\usepackage{cancel}
\usepackage{askmaps}
\usepackage{tabularx}
\usepackage[table]{xcolor}
%\usepackage{tikz}
\usepackage{multirow}
\usepackage{mathtools}
\usepackage{color, colortbl}
\usepackage{etoolbox}
\usepackage{pbox}
\usepackage{changepage}
\usepackage{xpatch}
\usepackage{array}
\usepackage{marvosym}
\usepackage{tabu}
\usepackage{multicol}
\usepackage{listings}
\usepackage{underscore}
\usepackage{filecontents}
\usepackage[]{algorithm2e}
\usepackage{ragged2e}

\newcolumntype{P}[1]{>{\centering\arraybackslash}m{#1}}
\definecolor{Gray}{gray}{0.75}
\definecolor{Gray2}{gray}{0.85}

\definecolor{lightBlue}{HTML}{DAE8FC}
\definecolor{Blue}{RGB}{51, 51, 204}

%\useinnertheme[lily]{rounded}
\usetheme{UniEvangelica}
%\usetheme{Copenhagen}
%\usetheme{Berlin}
%\usecolortheme{dolphin}
\tolerance=1
\emergencystretch=\maxdimen
\hyphenpenalty=10000
\hbadness=10000

\setbeamertemplate{navigation symbols}{}%remove navigation symbols


\let\olditem=\item% 
\renewcommand{\item}{\olditem \justifying}%
\def\center{\trivlist \centering\item\relax}
\def\endcenter{\endtrivlist}

\setbeamertemplate{itemize/enumerate body begin}{\large}
\setbeamertemplate{itemize/enumerate subbody begin}{\large}

\setbeamertemplate{itemize item}{\raisebox{0.1ex}{$\blacktriangleright$}\hskip0.1em}
\setbeamertemplate{itemize subitem}{\raisebox{0.1ex}{$\blacktriangleright$}\hskip0.1em}

\newcommand{\greenarrow}{\textcolor{green}{\rotatebox[origin=c]{180}{\MVArrowDown}}}

\newcommand{\redarrow}{\textcolor{red}{\MVArrowDown}}

%\newcommand{\ftable}{
%	\begin{table}
%		\large
%		\centering
%		\rowcolors{1}{\ifnumless{\rownum}{2}{Blue}{lightBlue}}{}
%}

\newenvironment{eftable}{
	\begin{table}
		\large
		\centering
		\rowcolors{1}{}{Blue}
		\rowcolors{1}{\ifnumless{\rownum}{2}{Blue}{lightBlue}}{}
	}
	{
	\end{table}
}


%\setbeamertemplate{frametitle}
%{
%	%\vspace*{-2em}	
%	\insertframetitle
%
%	 %\textcolor{white}{\LARGE \insertframetitle}
%
%}

% Specific definitions
\institute[]{\uppercase{Engenharia da Computação}}
\title[]{Prática em Fábrica de Software III}
\subtitle[]{\uppercase{Apresentação da Disciplina}}
\author[]{Prof. Alexandre Tannus}
\date{}

\AtBeginSection{\frame{\tableofcontents[currentsection]}}

\begin{document}
	\begin{frame}
		\titlepage
	\end{frame}
	
	\begin{frame}
		\frametitle{Professor Alexandre Tannus}
		\centering
			\huge Bacharel em Engenharia da Computação
			\begin{figure}
			\centering	\includegraphics[height=2cm, keepaspectratio]{../figs/puc.jpg} 			
			\end{figure}
			\huge Mestre em Engenharia Elétrica
			\begin{figure}
			\centering	\includegraphics[height=2cm, keepaspectratio]{../figs/ufmg} 		
			\end{figure}
	\end{frame}
	
	\begin{frame}[allowframebreaks]{Objetivos}
		\begin{itemize}

			\item Diferenciar circuitos analógicos e digitais
			\vspace{1em}
			\item Compreender os princípios da conversão analógico-digital
			\vspace{1em}
			\item Projetar circuitos simples de amplificação de sinais elétricos
			\vspace{1em}
			\item Avaliar os parâmetros de definição de sensores
			\vspace{1em}
			\item Identificar, formular e resolver problemas de engenharia.

		\end{itemize}
	\end{frame}

	\begin{frame}[allowframebreaks]{Ementa}
		\begin{itemize}
			\item Implementar soluções de problemas utilizando controladores lógico programáveis
			\vspace{1em}
			\item Revisar conceitos de eletrônica digital
			\item Sistemas eletrônicos e automação.
			\vspace{1em}
			\item Sistemas de controle.  
		\end{itemize}
	\end{frame}
	
	\begin{frame}
		\frametitle{Avaliações}
		\begin{eftable}
			\LARGE
			\begin{tabular}{c | c}
				\textcolor{white}{Avaliação} & 
				\textcolor{white}{Data} \\
				1 ${}^a$ V.A & 19/03/2020 \\
				2 ${}^a$ V.A & 07/05/2020 \\
				3 ${}^a$ V.A & 18/06/2020 \\			
			\end{tabular}
		\end{eftable}
	\end{frame}	
	
	\begin{frame}
		\frametitle{Trabalhos – Regras gerais}
		\begin{itemize}
			\item Data de entrega - Avaliação
			\begin{itemize}
				\item Entregue na data correta – 100%
				\item Entregue com até uma semana de atraso – 50%
				\item Entregue com mais de uma semana de atraso – 0%
			\end{itemize}
			\item Normas para os documentos
			\begin{itemize}
				\item \textbf{ABNT}
			\end{itemize}
			\item Plágio
			\begin{itemize}
				\item Em caso de plágio detectado em qualquer trabalho a nota será ZERO
			\end{itemize}
		\end{itemize}
	\end{frame}
	
	\begin{frame}
		\frametitle{Pontuação}
		\begin{itemize}
			\item As notas para correção de provas e trabalhos serão de 0,25 em 0,25 pontos. Os arredondamentos só serão feitos após a nota fechada para lançamento no sistema. 

			\item Nas provas: 
			\begin{itemize}
				\item erros absurdos - 0 pontos; 
				\item raciocínio parcialmente correto: 25\%, 50\% ou 75\% da questão; 
				\item acertos integrais ou com erros irrelevantes: 100\% .

			\end{itemize}

		\end{itemize}
	\end{frame}
	
	\begin{frame}
		\frametitle{Pontuação}
		\begin{center}
			\Huge \alert {Em caso de comprovação de cola em provas, trabalhos idênticos (inclusive com os mesmos erros) e plágios, a nota será \textbf{ZERADA}.}		
		\end{center}
	\end{frame}
	
	\begin{frame}
		\frametitle{Modelo de prova}
		\begin{itemize}
			\item Questões de ENADE e concursos públicos
			\vspace{1em}
			\item Questões objetivas e discursivas
			\vspace{1em}
			\item Conteúdos de aulas de laboratório poderão ser cobrados na prova
		\end{itemize}
	\end{frame}
	
%	\begin{frame}
%		\frametitle{Material da disciplina}
%		\begin{itemize}
%			\item Disponibilizado no Lyceum
%			\vspace{1em}
%			\item Disponível também no site \href{https://sites.google.com/site/professoralexandretannus}{\textcolor{blue}{https://sites.google.com/site/professoralexandretannus}}  
%		\end{itemize}
%	\end{frame}
	
	\begin{frame}
		\frametitle{Bibliografia Básica}
		\begin{itemize}
			\item MALVINO, A.; BATES, D.J. \textbf{Eletrônica – Volume II}, 8. ed., Porto Alegre, AMGH, 2016.
%			\vspace{1em} 
			\item SILVA, E.A. \textbf{Introdução às linguagens de programação para CLP}, São Paulo, Blucher, 2016.
			\vspace{1em} 
			\item STEVAN JUNIOR, S.L.; SILVA, R.A. \textbf{Automação e Instrumentação Industrial com Arduino: teoria e projetos} São Paulo, Érica, 2015
		\end{itemize}
	\end{frame}
	
	\begin{frame}[allowframebreaks]{Bibliografia Complementar}
		\begin{itemize}
			\item ALBUQUERQUE, R.O.; SEABRA, A.C. \textbf{Utilizando Eletrônica com AO, SCR, TRIAC, UJT, PUT, CI 555, LDR, LED, IGBT e FET de Potência} 2. ed., São Paulo: Érica, 2012

			\vspace{1em} 
			\item IDOETA, I.V.; CAPUANO, F.G. \textbf{Elementos de Eletrônica Digital.} 41 ed., São Paulo, Erica, 2012.
			\vspace{1em} 
			\item OLIVEIRA, A.S. \textbf{Sistemas Embarcados: hardware e o firmware na prática} 2.ed., São Paulo, Érica, 2010
			\vspace{1em} 
			\item OLIVEIRA, C.L.V.; ZANETTI, H.A.P. \textbf{Arduino descomplicado: como elaborar projetos de eletrônica} São Paulo, Érica, 2015.
			\vspace{1em} 

			\item THOMAZINI, D.; ALBUQUERQUE, P.U.B. \textbf{Sensores industriais: fundamentos e aplicações}, 8. ed., São Paulo, Érica, 2011.
		\end{itemize}
	\end{frame}
	
	\begin{frame}{}
	\end{frame}
\end{document}